% Created 2026-02-24 mar 08:54
% Intended LaTeX compiler: pdflatex
\documentclass[a4paper, 12pt, twoside, titlepage]{article}
\usepackage[utf8]{inputenc}
\usepackage[T1]{fontenc}
\usepackage{graphicx}
\usepackage{longtable}
\usepackage{wrapfig}
\usepackage{rotating}
\usepackage[normalem]{ulem}
\usepackage{amsmath}
\usepackage{amssymb}
\usepackage{capt-of}
\usepackage{hyperref}
\usepackage[spanish, es-tabla]{babel}
\usepackage{csquotes}
\usepackage[utf8]{inputenc}
\usepackage[T1]{fontenc}
\usepackage{a4wide}
\usepackage{tocloft}
\renewcommand{\cftsecleader}{\cftdotfill{\cftdotsep}}
\usepackage{fancyhdr}
\usepackage{graphicx}
\usepackage{wrapfig}
\usepackage{subcaption}
\usepackage{tikz}
\usetikzlibrary{shapes,arrows,matrix,fit}
\usepackage{circuitikz}
\pagestyle{fancy}
\cfoot{}
\fancyhead{}
\fancyhead[RE,LO]{\bfseries \nouppercase{\leftmark}}
\fancyhead[LE,RO]{\bfseries \thepage}
\usepackage{pdfpages}
\author{angruicam1}
\date{\today}
\title{}
\hypersetup{
 pdfauthor={angruicam1},
 pdftitle={},
 pdfkeywords={},
 pdfsubject={},
 pdfcreator={Emacs 27.1 (Org mode 9.5.2)}, 
 pdflang={Spanish}}
\usepackage[style=backend=biber,style=ieee]{biblatex}
\addbibresource{/home/angruicam1/Universidad/Termodinamica/Practica4 (Trabajo)/hvN.bib}
\begin{document}

\begin{titlepage}
\begin{center}
\includegraphics[scale=.5]{/home/angruicam1/Universidad/Materiales/Logos/logotipo-us-principal} \par
{\large \bf Universidad de Sevilla \par Doble Grado en Física y Matemáticas \par Termodinámica \par}
\vspace{2cm}
{\LARGE \bf Práctica 4 \par
 Determinación del incremento de entalpía de vaporización del nitrógeno \par}
\vspace{1cm}
{\large Ángel Ruiz Campos \par Grupo E2 \par}
\vspace{1cm}
{\large \today}
\end{center}
\vfill
\end{titlepage}

\tableofcontents

\newpage

\section{Introducción}
\label{sec:org80ec991}

El nitrógeno (N\(_2\)) líquido es estable, a presión atmosférica, hasta una
temperatura de \(77.35\) K (\(-195.8\) ºC) \autocite{Practicas2025,Lemmon2026}.
En contacto con el ambiente \(-\text{que se encuentra a una}\)
\(\text{temperatura superior}-\), el nitrógeno líquido absorbe calor del medio
y comienza a hervir de forma estacionaria. La tasa de evaporación del nitrógeno
líquido \(-\text{masa de N}_2 \text{ evaporado}\) \(\text{por unidad de tiempo}-\)
depende de las condiciones del contacto con el medio, por lo que se
almacena en recintos aislados térmicamente. La no existencia de paredes
adiabáticas perfectas implica que, en contacto con el ambiente, siempre
existirá evaporación.

Existe diversa bibliografía sobre el cálculo experimental de la
entalpía de vaporización del N\(_2\) mediante experiencias académicas. Esto se
debe a la bondad de sus resultados y a la simplicidad de sus montajes y
la facilidad de obtener comercialmente nitrógeno líquido \autocite{Knutson1969}.

Precisamente, el objeto de la presente memoria es describir la experiencia
presentada en \autocite{Practicas2025}, presentar y analizar sus resultados.
Este documento se enmarca en el trabajo académico desarrollado en la
asignatura \textquote{Termodinámica} del Doble Grado en Física y Matemáticas
de la Universidad de Sevilla.

Una forma de poder medir el incremento de entalpía de vaporización del N\(_2\)
se basa en emplear la técnica de termogravimetría, que consiste en medir el
cambio de masa por evaporación para deducir las energías que intervienen en
un proceso mediante los balances adecuados \autocite{Practicas2025}.

El principio de conservación de la energía relaciona la cantidad de calor
necesario para evaporar una masa de nitrógeno líquido con las pérdidas del
recipiente que lo contiene. Para ser capaz de determinar ambas magnitudes, es
necesario generar distintas condiciones experimentales introduciendo un
trabajo disipativo. De esta forma, el balance de energías por unidad de tiempo
vendría dado por la ecuación:

\begin{equation}
\label{eq:orgb37ddea}
\dot{m}\Delta h = \dot{Q} - \dot{W},
\end{equation}
donde \(\dot{W}\) es la potencia disipada, \(\dot{Q}\) son las pérdidas por unidad
de tiempo, \(\dot{m}\) es la tasa de evaporación y \(\Delta h\) es el incremento
de entalpía de vaporización del N\(_2\) \autocite{Practicas2025}.

La bondad del resultado experimental obtenido se puede determinar tomando
como referencia el valor del incremento de entalpía de vaporización del
N\(_2\) igual a \(199.176\) J g\(^{-1}\) \autocite{Lemmon2026}.

\section{Dispositivos experimentales}
\label{sec:org8f51deb}

El montaje experimental de la siguiente experiencia se ha realizado en el
Laboratorio de Termodinámica de la Facultad de Física de la Universidad de
Sevilla. Se ha dispuesto de un vaso Dewar con nitrógeno líquido, de un
resistor conectado a una fuente de tensión y de un granatario, un cronómetro,
un amperímetro y un voltímetro como instrumentos de medida. La figura
\ref{org3375f91} muestra el montaje empleado.

\begin{figure}[h!]
\begin{center}

\begin{subfigure}{0.2\textwidth}
\includegraphics[width=0.9\linewidth, height=6cm]{dewar}
\end{subfigure}
\begin{subfigure}{0.4\textwidth}
\includegraphics[width=0.9\linewidth, height=6cm]{circuito}
\end{subfigure}

\end{center}
\caption{\label{org3375f91}Fotografías del montaje experimental tomadas el 19 de noviembre de 2025 por el autor. A la izquierda se aprecia el vaso Dewar sobre el granatario para la medición de la disminución de la masa debida a la evaporación del N\(_2\) y su escape del recinto \(-\text{se puede apreciar por la escarcha formada en los tubos que salen del vaso Dewar}-\). A la derecha puede observase el cronómetro y la fuente de tensión, el amperímetro y el voltímetro conectados a un circuito que alimenta el resistor. La figura \ref{org81d2afa} detalla el esquema del circuito.}
\end{figure}

\begin{figure}[h!]
\begin{center}
  \ctikzset{bipoles/thickness=0.85}
  \begin{circuitikz}[american]
     \draw (0,0) node[below]{\small $6$} to[R, *-*] (0,4) node[above]{\small $7$} -- (1,4)
     to[spst, *-*] (6,4) node[right]{\small $1$}
     to[isource, *-*] (6,0) node[right]{\small $2$} -- (4.5,0) node[below]{\small $4$}
     to[rmeterwa, t=A, *-*] (2.5,0) node[below]{\small $5$} -- (0,0);
     \draw (1,0) node[below]{\small $3$} to[rmeterwa, t=V, *-*] (1,4) node[above]{\small $8$};
  \end{circuitikz}
\end{center}
\caption{\label{org81d2afa}Esquema del circuito empleado en la práctica de laboratorio \(-\text{reproducido de}\) \autocite{Practicas2025}\(-\). El montaje está conformado por una fuente de tensión conectada con un resistor, de un interruptor para abrir o cerrar el circuito y de un amperímetro conectado en serie y un voltímetro en paralelo.}
\end{figure}

La experiencia comienza encendiendo el granatario y configurando el
amperímetro y el voltímetro para medir en corriente alterna. A continuación,
se enciende la fuente de tensión y se conmuta el interruptor del circuito.

Preparado el instrumental, se hace circular una corriente de \(0.45\) A
por el circuito. En ese instante, se anota la lectura del granatario
\(-\text{que será la masa inicial de nitrógeno líquido}-\) y se pone en marcha
el cronómetro. Durante \(15\) min se anotan las lecturas del granatario, del
voltímetro y del amperímetro cada \(30\) s.

La experiencia se repite para valores de intensidad de \(0.4\) A, \(0.3\) A y
\(0.25\) A. Finalmente se desconecta el circuito y se mide el comportamiento
libre del sistema. Entre cada una de las experiencias se deja pasar un tiempo
para que la evaporación del nitrógeno líquido alcance un ritmo estacionario.
La figura \ref{org399278c} resume el procedimiento mediante un diagrama de flujo.

\begin{figure}[h!]
\begin{center}
\begin{tikzpicture}[auto]
  \tikzstyle{decision} = [diamond, draw=blue, thick, fill=blue!20, 
      text width=6em, text badly centered, inner sep=1pt, aspect=2];
  \tikzstyle{block} = [rectangle, draw=blue, thick, fill=blue!20,
      text width=7em, text badly centered, minimum height=4em];
  \tikzstyle{line} = [draw, thick, -latex'];
  \tikzstyle{cloud} = [draw=red, thick, ellipse, fill=red!20,
      text width=6em, text badly centered, minimum width=4.2cm];
  \matrix[column sep=5mm,row sep=7mm]
  {
      \node [cloud] (11) {Vaso Dewar con N$_2$ líquido}; & & \\

      \node [block] (21) {Encender la fuente de corriente: $I = 0.45$ A}; & &
      \node [block] (22) {\small Disminuir la intensidad: \\ 1.º $0.40$ A \\ 2.º $0.30$ A \\ 3.º $0.25$ A \\ 4.º $\phantom{0,0}0$ A}; \\

      \node [decision] (31) {¿$t = 15$ min?}; & &
      \node [decision] (32) {¿$I = 0$ A?}; \\

      \node [block] (41) {Esperar $30$ s}; & & 
      \node [cloud] (42) {\small Ajustes: \\ $m$ frente a $t$ \\ $\dot{m}$ frente a $\dot{W}$}; \\

      \node [block] (51) {Medir los valores de $m$, $I$ y $V$}; & & \\
  };
  \tikzstyle{every path}=[line]
  \path (11) -- (21);
  \path (21) -- (31);
  \path (31) -- node {No} (41);
  \path (41) -- (51);
  \path (51) -| +(-2,2) |- (31);
  \path (22) -- (31);
  \path (31) -- node {Sí} (32);
  \path (32) -- node {No} (22);
  \path (32) -- node {Sí} (42);
\end{tikzpicture}
\end{center}
\caption{\label{org399278c}Diagrama de flujo que esquematiza el procedimiento de la experiencia. En rojo se representan el sistema objeto de estudio y los resultados a analizar. Los rombos simbolizan decisiones y los cuadrados acciones.}
\end{figure}

\section{Resultados}
\label{sec:org7ed5928}

Para la presentación de los datos y su tratamiento se han seguido los
criterios expuestos en \autocite{Olalla2025} y \autocite{TEB2025}. El ajuste de los
datos y su representación gráfica se ha realizado empleando el \emph{software} libre
\emph{gnuplot} \autocite{gnuplot54p2}, mientras que para la redacción de la
presente memoria se ha hecho uso del modo de edición \emph{org mode} del editor
de texto \emph{Emacs} \autocite{orgmode952}. La generación de las figuras \ref{org81d2afa}
y \ref{org399278c} ha sido realizada mediante la librería \(\text{Ti}k\texk{Z}\) de
\(\LaTeX\). Para el caso del circuito, en particular, se ha hecho uso de la
librería \(\text{CircuiTi}k\text{Z}\). Todos los archivos empleados para la
compilación de esta memoria se encuentran accesibles en el repositorio de
GitHub con enlace \url{https://github.com/angruicam/Termodinamica}. En particular,
puede accederse al fichero con los datos obtenidos en el laboratorio.

La toma en el laboratorio de los datos recogidos en esta memoria fue realizada
el día 19 de noviembre de 2025. Siguiendo el procedimiento descrito, se
obtuvieron cinco series de datos, correspondientes a las corrientes de \(0.45\)
A, \(0.4\) A, \(0.3\) A, \(0.25\) A y al comportamiento libre del sistema. Las tablas
\ref{tab:orgcca50fe} y \ref{tab:org7d23fbf} muestran los resultados medidos. En la cabecera de cada
columna se indica la incertidumbre de los datos, que a su vez se corresponde
con una unidad del último dígito de la lectura del aparato de medida. En el
caso del tiempo, aunque el cronómetro mide hasta las centésimas de segundo se
toma como incertidumbre \(\pm 1\) s debido al tiempo de reacción.

A partir de las medidas realizadas se obtiene el valor de la tasa de
evaporación del nitrógeno líquido, \(\dot{m}\), para cada serie de datos. En
primer lugar, la tabla \ref{tab:org6936e7d} presenta los valores calculados para la masa total
de N\(_2\) evaporado en cada medición, que se obtiene como la diferencia entre
la medida inicial marcada en el granatario (\(1520.8\) g) y cada una de las
ulteriores medidas. Su incertidumbre se obtiene, por tanto, como la suma de las
incertidumbres de ambas magnitudes. En la figura \ref{orgba77be9}
se representa la masa de N\(_2\) evaporada frente al tiempo para cada una de
las series de datos. Del ajuste lineal de cada conjunto de medidas se deriva
el valor de la tasa de evaporación correspondiente a cada condición
experimental. Pueden extraerse directamente del archivo \hyperref[org195eda0]{\emph{fit.log}} generado
por \emph{gnuplot}.

\phantomsection
\label{org195eda0}
\begin{verbatim}
Final set of parameters            Asymptotic Standard Error
=======================            ==========================
m45             = 0.0270726        +/- 0.000169     (0.6243%)
m45i            = 1.0625           +/- 0.08855      (8.334%)
m40             = 0.0220524        +/- 2.923e-05    (0.1325%)
m40i            = 26.128           +/- 0.01531      (0.0586%)
m30             = 0.0169382        +/- 3.109e-05    (0.1835%)
m30i            = 47.3875          +/- 0.01629      (0.03437%)
m25             = 0.0153871        +/- 2.29e-05     (0.1488%)
m25i            = 63.3565          +/- 0.012        (0.01893%)
m0              = 0.0106801        +/- 5.118e-05    (0.4792%)
m0i             = 77.4101          +/- 0.02681      (0.03464%)
\end{verbatim}

\begin{table}[htbp]
\centering
\begin{tabular}{|c||c|c|c||c|c|c|}
\hline
\(t/\) s & \(m_{45}/\) g & \(V_{45}/\) V & \(I_{45}/\) A & \(m_{40}/\) g & \(V_{40}/\) V & \(I_{40}/\) A\\
\(\pm\, 1\) & \(\pm\, 0.1\) & \(\pm\, 0.001\) & \(\pm\, 0.01\) & \(\pm\, 0.1\) & \(\pm\, 0.001\) & \(\pm\, 0.01\)\\
\hline
0 & \(1520.8\) & \(5.971\) & \(0.45\) & \(1494.7\) & \(5.334\) & \(0.40\)\\
30 & \(1519.2\) & \(5.991\) & \(0.45\) & \(1494.0\) & \(5.335\) & \(0.40\)\\
60 & \(1518.2\) & \(5.998\) & \(0.45\) & \(1493.3\) & \(5.332\) & \(0.40\)\\
90 & \(1517.3\) & \(5.992\) & \(0.46\) & \(1492.8\) & \(5.326\) & \(0.40\)\\
120 & \(1516.5\) & \(6.006\) & \(0.45\) & \(1492.1\) & \(5.325\) & \(0.40\)\\
150 & \(1515.5\) & \(6.007\) & \(0.46\) & \(1491.3\) & \(5.338\) & \(0.40\)\\
180 & \(1514.8\) & \(6.015\) & \(0.45\) & \(1490.7\) & \(5.333\) & \(0.40\)\\
210 & \(1513.9\) & \(6.055\) & \(0.46\) & \(1490.0\) & \(5.325\) & \(0.40\)\\
240 & \(1513.1\) & \(6.048\) & \(0.46\) & \(1489.4\) & \(5.341\) & \(0.40\)\\
270 & \(1512.2\) & \(6.055\) & \(0.46\) & \(1488.7\) & \(5.343\) & \(0.40\)\\
300 & \(1511.4\) & \(6.031\) & \(0.46\) & \(1488.1\) & \(5.329\) & \(0.40\)\\
330 & \(1510.6\) & \(6.035\) & \(0.46\) & \(1487.4\) & \(5.334\) & \(0.40\)\\
360 & \(1509.8\) & \(6.042\) & \(0.46\) & \(1486.7\) & \(5.318\) & \(0.40\)\\
390 & \(1509.0\) & \(6.043\) & \(0.46\) & \(1486.1\) & \(5.330\) & \(0.41\)\\
420 & \(1508.2\) & \(6.052\) & \(0.46\) & \(1485.4\) & \(5.342\) & \(0.41\)\\
450 & \(1507.4\) & \(6.039\) & \(0.46\) & \(1484.7\) & \(5.330\) & \(0.41\)\\
480 & \(1506.6\) & \(6.043\) & \(0.46\) & \(1484.1\) & \(5.365\) & \(0.40\)\\
510 & \(1505.8\) & \(6.032\) & \(0.46\) & \(1483.4\) & \(5.364\) & \(0.41\)\\
540 & \(1505.0\) & \(6.024\) & \(0.45\) & \(1482.7\) & \(5.345\) & \(0.40\)\\
570 & \(1504.2\) & \(6.019\) & \(0.46\) & \(1482.1\) & \(5.364\) & \(0.41\)\\
600 & \(1503.4\) & \(6.028\) & \(0.45\) & \(1481.4\) & \(5.355\) & \(0.41\)\\
630 & \(1502.6\) & \(6.004\) & \(0.45\) & \(1480.8\) & \(5.342\) & \(0.41\)\\
660 & \(1501.9\) & \(6.024\) & \(0.46\) & \(1480.1\) & \(5.349\) & \(0.41\)\\
690 & \(1501.1\) & \(6.014\) & \(0.46\) & \(1479.4\) & \(5.370\) & \(0.41\)\\
720 & \(1500.3\) & \(6.026\) & \(0.46\) & \(1478.8\) & \(5.364\) & \(0.41\)\\
750 & \(1499.5\) & \(6.044\) & \(0.46\) & \(1478.1\) & \(5.382\) & \(0.41\)\\
780 & \(1498.7\) & \(6.043\) & \(0.46\) & \(1477.5\) & \(5.386\) & \(0.41\)\\
810 & \(1497.9\) & \(6.043\) & \(0.46\) & \(1476.8\) & \(5.376\) & \(0.41\)\\
840 & \(1497.2\) & \(6.053\) & \(0.46\) & \(1476.2\) & \(5.342\) & \(0.41\)\\
870 & \(1496.4\) & \(6.044\) & \(0.46\) & \(1475.5\) & \(5.353\) & \(0.40\)\\
900 & \(1495.7\) & \(6.032\) & \(0.46\) & \(1474.9\) & \(5.365\) & \(0.41\)\\
\hline
\end{tabular}
\caption{\label{tab:orgcca50fe}Series de datos medidos en el laboratorio para las corrientes de \(0.45\) A y \(0.4\) A. El subíndice se relaciona con las centésimas de amperio de la serie correspondiente. Los valores de las masas expuestos se corresponden con las medidas del granatario, y no directamente con la masa de nitrógeno líquido evaporada. Dichos valores se obtienen en la tabla \ref{tab:org6936e7d}.}

\end{table}

\begin{table}[htbp]
\centering
\begin{tabular}{|c||c|c|c||c|c|c||c|}
\hline
\(t/\) s & \(m_{30}/\) g & \(V_{30}/\) V & \(I_{30}/\) A & \(m_{25}/\) g & \(V_{25}/\) V & \(I_{25}/\) A & \(m_0/\) g\\
\(\pm\, 1\) & \(\pm\, 0.1\) & \(\pm\, 0.001\) & \(\pm\, 0.01\) & \(\pm\, 0.1\) & \(\pm\, 0.001\) & \(\pm\, 0.01\) & \(\pm\, 0.1\)\\
\hline
0 & \(1473.3\) & \(3.977\) & \(0.30\) & \(1457.4\) & \(3.388\) & \(0.25\) & \(1443.2\)\\
30 & \(1472.9\) & \(3.972\) & \(0.30\) & \(1457.0\) & \(3.398\) & \(0.25\) & \(1442.9\)\\
60 & \(1472.4\) & \(3.959\) & \(0.30\) & \(1456.5\) & \(3.403\) & \(0.25\) & \(1442.7\)\\
90 & \(1471.9\) & \(3.967\) & \(0.30\) & \(1456.1\) & \(3.379\) & \(0.25\) & \(1442.4\)\\
120 & \(1471.4\) & \(3.980\) & \(0.30\) & \(1455.6\) & \(3.401\) & \(0.25\) & \(1442.1\)\\
150 & \(1470.9\) & \(3.983\) & \(0.30\) & \(1455.2\) & \(3.406\) & \(0.25\) & \(1441.8\)\\
180 & \(1470.4\) & \(3.972\) & \(0.30\) & \(1454.7\) & \(3.395\) & \(0.25\) & \(1441.5\)\\
210 & \(1469.9\) & \(3.962\) & \(0.30\) & \(1454.2\) & \(3.397\) & \(0.25\) & \(1441.2\)\\
240 & \(1469.4\) & \(3.971\) & \(0.30\) & \(1453.8\) & \(3.407\) & \(0.25\) & \(1440.9\)\\
270 & \(1468.9\) & \(3.972\) & \(0.30\) & \(1453.3\) & \(3.425\) & \(0.25\) & \(1440.5\)\\
300 & \(1468.3\) & \(3.976\) & \(0.30\) & \(1452.8\) & \(3.423\) & \(0.25\) & \(1440.2\)\\
330 & \(1467.9\) & \(3.982\) & \(0.30\) & \(1452.3\) & \(3.425\) & \(0.25\) & \(1439.9\)\\
360 & \(1467.3\) & \(3.995\) & \(0.30\) & \(1451.9\) & \(3.423\) & \(0.25\) & \(1439.6\)\\
390 & \(1466.8\) & \(3.992\) & \(0.30\) & \(1451.4\) & \(3.429\) & \(0.25\) & \(1439.3\)\\
420 & \(1466.3\) & \(3.975\) & \(0.30\) & \(1451.0\) & \(3.415\) & \(0.25\) & \(1439.0\)\\
450 & \(1465.8\) & \(3.970\) & \(0.30\) & \(1450.5\) & \(3.393\) & \(0.25\) & \(1438.7\)\\
480 & \(1465.2\) & \(3.972\) & \(0.30\) & \(1450.0\) & \(3.398\) & \(0.25\) & \(1438.3\)\\
510 & \(1464.7\) & \(3.960\) & \(0.30\) & \(1449.6\) & \(3.392\) & \(0.25\) & \(1438.0\)\\
540 & \(1464.2\) & \(3.980\) & \(0.30\) & \(1449.1\) & \(3.384\) & \(0.25\) & \(1437.7\)\\
570 & \(1463.7\) & \(3.973\) & \(0.30\) & \(1448.7\) & \(3.381\) & \(0.25\) & \(1437.4\)\\
600 & \(1463.3\) & \(3.983\) & \(0.30\) & \(1448.2\) & \(3.388\) & \(0.25\) & \(1437.0\)\\
630 & \(1462.7\) & \(3.928\) & \(0.30\) & \(1447.8\) & \(3.386\) & \(0.25\) & \(1436.7\)\\
660 & \(1462.3\) & \(3.959\) & \(0.30\) & \(1447.3\) & \(3.383\) & \(0.25\) & \(1436.3\)\\
690 & \(1461.7\) & \(3.986\) & \(0.30\) & \(1446.8\) & \(3.385\) & \(0.25\) & \(1436.0\)\\
720 & \(1461.2\) & \(3.979\) & \(0.30\) & \(1446.4\) & \(3.383\) & \(0.25\) & \(1435.7\)\\
750 & \(1460.7\) & \(3.979\) & \(0.30\) & \(1445.9\) & \(3.381\) & \(0.25\) & \(1435.4\)\\
780 & \(1460.2\) & \(3.979\) & \(0.30\) & \(1445.4\) & \(3.359\) & \(0.25\) & \(1435.0\)\\
810 & \(1459.7\) & \(3.986\) & \(0.30\) & \(1445.0\) & \(3.371\) & \(0.25\) & \(1434.7\)\\
840 & \(1459.2\) & \(3.972\) & \(0.30\) & \(1444.5\) & \(3.374\) & \(0.25\) & \(1434.3\)\\
870 & \(1458.7\) & \(3.987\) & \(0.30\) & \(1444.1\) & \(3.362\) & \(0.25\) & \(1434.0\)\\
900 & \(1458.2\) & \(3.983\) & \(0.30\) & \(1443.6\) & \(3.358\) & \(0.25\) & \(1433.7\)\\
\hline
\end{tabular}
\caption{\label{tab:org7d23fbf}Series de datos medidos en el laboratorio para las corrientes de \(0.3\) A, \(0.25\) A y el comportamiento libre del sistema. El subíndice se relaciona con las centésimas de amperio de la serie correspondiente. Los valores de las masas expuestos se corresponden con las medidas del granatario, y no directamente con la masa de nitrógeno líquido evaporada. Dichos valores se obtienen en la tabla \ref{tab:org6936e7d}.}

\end{table}

\begin{table}[htbp]
\centering
\begin{tabular}{|c|c|c|c|c|c|}
\hline
\(t/\) s & \(m^v_{45}/\) g & \(m^v_{40}/\) g & \(m^v_{30}/\) g & \(m^v_{25}/\) g & \(m^v_0/\) g\\
\(\pm\, 1\) & \(\pm\, 0.2\) & \(\pm\, 0.2\) & \(\pm\, 0.2\) & \(\pm\, 0.2\) & \(\pm\, 0.2\)\\
\hline
0 & \(0\) & \(26.1\) & \(47.5\) & \(63.4\) & \(77.6\)\\
30 & \(1.6\) & \(26.8\) & \(47.9\) & \(63.8\) & \(77.9\)\\
60 & \(2.6\) & \(27.5\) & \(48.4\) & \(64.3\) & \(78.1\)\\
90 & \(3.5\) & \(28.0\) & \(48.9\) & \(64.7\) & \(78.4\)\\
120 & \(4.3\) & \(28.7\) & \(49.4\) & \(65.2\) & \(78.7\)\\
150 & \(5.3\) & \(29.5\) & \(49.9\) & \(65.6\) & \(79.0\)\\
180 & \(6.0\) & \(30.1\) & \(50.4\) & \(66.1\) & \(79.3\)\\
210 & \(6.9\) & \(30.8\) & \(50.9\) & \(66.6\) & \(79.6\)\\
240 & \(7.7\) & \(31.4\) & \(51.4\) & \(67.0\) & \(79.9\)\\
270 & \(8.6\) & \(32.1\) & \(51.9\) & \(67.5\) & \(80.3\)\\
300 & \(9.4\) & \(32.7\) & \(52.5\) & \(68.0\) & \(80.6\)\\
330 & \(10.2\) & \(33.4\) & \(52.9\) & \(68.5\) & \(80.9\)\\
360 & \(11.0\) & \(34.1\) & \(53.5\) & \(68.9\) & \(81.2\)\\
390 & \(11.8\) & \(34.7\) & \(54.0\) & \(69.4\) & \(81.5\)\\
420 & \(12.6\) & \(35.4\) & \(54.5\) & \(69.8\) & \(81.8\)\\
450 & \(13.4\) & \(36.1\) & \(55.0\) & \(70.3\) & \(82.1\)\\
480 & \(14.2\) & \(36.7\) & \(55.6\) & \(70.8\) & \(82.5\)\\
510 & \(15.0\) & \(37.4\) & \(56.1\) & \(71.2\) & \(82.8\)\\
540 & \(15.8\) & \(38.1\) & \(56.6\) & \(71.7\) & \(83.1\)\\
570 & \(16.6\) & \(38.7\) & \(57.1\) & \(72.1\) & \(83.4\)\\
600 & \(17.4\) & \(39.4\) & \(57.5\) & \(72.6\) & \(83.8\)\\
630 & \(18.2\) & \(40.0\) & \(58.1\) & \(73.0\) & \(84.1\)\\
660 & \(18.9\) & \(40.7\) & \(58.5\) & \(73.5\) & \(84.5\)\\
690 & \(19.7\) & \(41.4\) & \(59.1\) & \(74.0\) & \(84.8\)\\
720 & \(20.5\) & \(42.0\) & \(59.6\) & \(74.4\) & \(85.1\)\\
750 & \(21.3\) & \(42.7\) & \(60.1\) & \(74.9\) & \(85.4\)\\
780 & \(22.1\) & \(43.3\) & \(60.6\) & \(75.4\) & \(85.8\)\\
810 & \(22.9\) & \(44.0\) & \(61.1\) & \(75.8\) & \(86.1\)\\
840 & \(23.6\) & \(44.6\) & \(61.6\) & \(76.3\) & \(86.5\)\\
870 & \(24.4\) & \(45.3\) & \(62.1\) & \(76.7\) & \(86.8\)\\
900 & \(25.1\) & \(45.9\) & \(62.6\) & \(77.2\) & \(87.1\)\\
\hline
\end{tabular}
\caption{\label{tab:org6936e7d}Valores de la masa de nitrógeno líquido evaporado para cada intervalo de tiempo. El subíndice se relaciona con las centésimas de amperio de la serie correspondiente y el superíndice indica que es la masa de N\(_2\) que se ha evaporado.}

\end{table}

\begin{figure}
\begin{center}
  % GNUPLOT: LaTeX picture with Postscript
\begingroup
  \makeatletter
  \providecommand\color[2][]{%
    \GenericError{(gnuplot) \space\space\space\@spaces}{%
      Package color not loaded in conjunction with
      terminal option `colourtext'%
    }{See the gnuplot documentation for explanation.%
    }{Either use 'blacktext' in gnuplot or load the package
      color.sty in LaTeX.}%
    \renewcommand\color[2][]{}%
  }%
  \providecommand\includegraphics[2][]{%
    \GenericError{(gnuplot) \space\space\space\@spaces}{%
      Package graphicx or graphics not loaded%
    }{See the gnuplot documentation for explanation.%
    }{The gnuplot epslatex terminal needs graphicx.sty or graphics.sty.}%
    \renewcommand\includegraphics[2][]{}%
  }%
  \providecommand\rotatebox[2]{#2}%
  \@ifundefined{ifGPcolor}{%
    \newif\ifGPcolor
    \GPcolortrue
  }{}%
  \@ifundefined{ifGPblacktext}{%
    \newif\ifGPblacktext
    \GPblacktextfalse
  }{}%
  % define a \g@addto@macro without @ in the name:
  \let\gplgaddtomacro\g@addto@macro
  % define empty templates for all commands taking text:
  \gdef\gplbacktext{}%
  \gdef\gplfronttext{}%
  \makeatother
  \ifGPblacktext
    % no textcolor at all
    \def\colorrgb#1{}%
    \def\colorgray#1{}%
  \else
    % gray or color?
    \ifGPcolor
      \def\colorrgb#1{\color[rgb]{#1}}%
      \def\colorgray#1{\color[gray]{#1}}%
      \expandafter\def\csname LTw\endcsname{\color{white}}%
      \expandafter\def\csname LTb\endcsname{\color{black}}%
      \expandafter\def\csname LTa\endcsname{\color{black}}%
      \expandafter\def\csname LT0\endcsname{\color[rgb]{1,0,0}}%
      \expandafter\def\csname LT1\endcsname{\color[rgb]{0,1,0}}%
      \expandafter\def\csname LT2\endcsname{\color[rgb]{0,0,1}}%
      \expandafter\def\csname LT3\endcsname{\color[rgb]{1,0,1}}%
      \expandafter\def\csname LT4\endcsname{\color[rgb]{0,1,1}}%
      \expandafter\def\csname LT5\endcsname{\color[rgb]{1,1,0}}%
      \expandafter\def\csname LT6\endcsname{\color[rgb]{0,0,0}}%
      \expandafter\def\csname LT7\endcsname{\color[rgb]{1,0.3,0}}%
      \expandafter\def\csname LT8\endcsname{\color[rgb]{0.5,0.5,0.5}}%
    \else
      % gray
      \def\colorrgb#1{\color{black}}%
      \def\colorgray#1{\color[gray]{#1}}%
      \expandafter\def\csname LTw\endcsname{\color{white}}%
      \expandafter\def\csname LTb\endcsname{\color{black}}%
      \expandafter\def\csname LTa\endcsname{\color{black}}%
      \expandafter\def\csname LT0\endcsname{\color{black}}%
      \expandafter\def\csname LT1\endcsname{\color{black}}%
      \expandafter\def\csname LT2\endcsname{\color{black}}%
      \expandafter\def\csname LT3\endcsname{\color{black}}%
      \expandafter\def\csname LT4\endcsname{\color{black}}%
      \expandafter\def\csname LT5\endcsname{\color{black}}%
      \expandafter\def\csname LT6\endcsname{\color{black}}%
      \expandafter\def\csname LT7\endcsname{\color{black}}%
      \expandafter\def\csname LT8\endcsname{\color{black}}%
    \fi
  \fi
    \setlength{\unitlength}{0.0500bp}%
    \ifx\gptboxheight\undefined%
      \newlength{\gptboxheight}%
      \newlength{\gptboxwidth}%
      \newsavebox{\gptboxtext}%
    \fi%
    \setlength{\fboxrule}{0.5pt}%
    \setlength{\fboxsep}{1pt}%
    \definecolor{tbcol}{rgb}{1,1,1}%
\begin{picture}(7200.00,10080.00)%
    \gplgaddtomacro\gplbacktext{%
      \csname LTb\endcsname%%
      \put(-66,2904){\makebox(0,0)[r]{\strut{}$0$}}%
      \csname LTb\endcsname%%
      \put(-66,3990){\makebox(0,0)[r]{\strut{}$15$}}%
      \csname LTb\endcsname%%
      \put(-66,5075){\makebox(0,0)[r]{\strut{}$30$}}%
      \csname LTb\endcsname%%
      \put(-66,6161){\makebox(0,0)[r]{\strut{}$45$}}%
      \csname LTb\endcsname%%
      \put(-66,7247){\makebox(0,0)[r]{\strut{}$60$}}%
      \csname LTb\endcsname%%
      \put(-66,8332){\makebox(0,0)[r]{\strut{}$75$}}%
      \csname LTb\endcsname%%
      \put(-66,9418){\makebox(0,0)[r]{\strut{}$90$}}%
      \csname LTb\endcsname%%
      \put(66,2684){\makebox(0,0){\strut{}$0$}}%
      \csname LTb\endcsname%%
      \put(949,2684){\makebox(0,0){\strut{}$120$}}%
      \csname LTb\endcsname%%
      \put(1833,2684){\makebox(0,0){\strut{}$240$}}%
      \csname LTb\endcsname%%
      \put(2716,2684){\makebox(0,0){\strut{}$360$}}%
      \csname LTb\endcsname%%
      \put(3600,2684){\makebox(0,0){\strut{}$480$}}%
      \csname LTb\endcsname%%
      \put(4483,2684){\makebox(0,0){\strut{}$600$}}%
      \csname LTb\endcsname%%
      \put(5366,2684){\makebox(0,0){\strut{}$720$}}%
      \csname LTb\endcsname%%
      \put(6250,2684){\makebox(0,0){\strut{}$840$}}%
      \csname LTb\endcsname%%
      \put(7133,2684){\makebox(0,0){\strut{}$960$}}%
    }%
    \gplgaddtomacro\gplfronttext{%
      \csname LTb\endcsname%%
      \put(-539,6161){\rotatebox{-270}{\makebox(0,0){\strut{}Masa de N$_2$ evaporado, $m^v$/ g}}}%
      \put(3599,2354){\makebox(0,0){\strut{}\shortstack{Tiempo, $t$/ s\\ \,}}}%
      \csname LTb\endcsname%%
      \put(1057,283){\makebox(0,0)[l]{\strut{}Datos experimentales ($I = 0.45$A), $R^2 = 0.9989$}}%
      \csname LTb\endcsname%%
      \put(1057,723){\makebox(0,0)[l]{\strut{}Datos experimentales ($I = 0.40$A), $R^2 = 0.99995$}}%
      \csname LTb\endcsname%%
      \put(1057,1163){\makebox(0,0)[l]{\strut{}Datos experimentales ($I = 0.30$A), $R^2 = 0.99990$}}%
      \csname LTb\endcsname%%
      \put(1057,1603){\makebox(0,0)[l]{\strut{}Datos experimentales ($I = 0.25$A), $R^2 = 0.99994$}}%
      \csname LTb\endcsname%%
      \put(1057,2043){\makebox(0,0)[l]{\strut{}Datos experimentales ($I = 0$A), $R^2 = 0.9993$}}%
      \colorrgb{0.00,0.00,1.00}%%
      \put(1833,3773){\rotatebox{14}{\makebox(0,0)[l]{\strut{}$m^v_{45}/$ g $= 1.1$$\, +\, 2.71e-02\, t/$ s}}}%
      \colorrgb{0.00,0.39,0.00}%%
      \put(1833,5437){\rotatebox{12}{\makebox(0,0)[l]{\strut{}$m^v_{40}/$ g $= 26.1$$\, +\, 2.21e-02\, t/$ s}}}%
      \colorrgb{0.75,0.00,1.00}%%
      \put(1833,6885){\rotatebox{9}{\makebox(0,0)[l]{\strut{}$m^v_{30}/$ g $= 47.4$$\, +\, 1.69e-02\, t/$ s}}}%
      \colorrgb{0.75,0.25,0.00}%%
      \put(1833,8043){\rotatebox{8}{\makebox(0,0)[l]{\strut{}$m^v_{25}/$ g $= 63.4$$\, +\, 1.54e-02\, t/$ s}}}%
      \colorrgb{1.00,0.00,0.00}%%
      \put(1833,8911){\rotatebox{6}{\makebox(0,0)[l]{\strut{}$m^v_0/$ g $= 77.4$$\, +\, 1.07e-02\, t/$ s}}}%
      \csname LTb\endcsname%%
      \put(3599,9748){\makebox(0,0){\strut{}\bf Diagrama termogravimétrico del N_2}}%
    }%
    \gplbacktext
    \put(0,0){\includegraphics[width={360.00bp},height={504.00bp}]{diagramaTermogravimetrico}}%
    \gplfronttext
  \end{picture}%
\endgroup

\end{center}
\caption{\label{orgba77be9}Diagrama termogravimétrico obtenido a partir de la evaporación del nitrógeno líquido bajo distintas condiciones experimentales. Sobre cada recta de ajuste se presenta su ecuación, cuya pendiente equivale a la tasa de evaporación del nitrógeno líquido al hacer circular por el circuito la corriente correspondiente a cada serie de datos.}
\end{figure}

Por su parte, conforme a la ley de Watt, la potencia disipada por el resistor
viene dada por la expresión:
\begin{equation}
\label{eq:org76ab069}
\dot{W} = - V\,I,
\end{equation}
donde \(I\) y \(V\) se corresponden, para cada condición experimental generada,
con la media de los valores tabulados en las tablas \ref{tab:orgcca50fe} y \ref{tab:org7d23fbf}.
Como todas las medidas del amperímetro y el voltímetro tienen la misma
incertidumbre, también será la incertidumbre de su valor medio. Mediante
propagación de incertidumbres se llega a:
\begin{equation*}
\Delta\dot{W} = V\,\Delta I + I\,\Delta V.
\end{equation*}

La tabla \ref{tab:orgc2c24bf} muestra los valores de la tasa de evaporación, \(\dot{m}\),
y la potencia disipada, \(\dot{W}\), para cada experiencia.

\begin{table}[htbp]
\centering
\begin{tabular}{|c|c|c|}
\hline
Condición experimental & \(\dot{m}/10^{-2}\) g s\(^{-1}\) & \(\dot{W}/\) J s\(^{-1}\)\\
\hline
\(I = 0.45\) A & \(2.707 \pm 0.017\) & \(-2.76 \pm 0.06\)\\
\(I = 0.40\) A & \(2.205 \pm 0.003\) & \(-2.16 \pm 0.05\)\\
\(I = 0.30\) A & \(1.694 \pm 0.003\) & \(-1.19 \pm 0.04\)\\
\(I = 0.25\) A & \(1.539 \pm 0.002\) & \(-0.85 \pm 0.03\)\\
\(I = 0\) A & \(1.068 \pm 0.005\) & \(0\)\\
\hline
\end{tabular}
\caption{\label{tab:orgc2c24bf}Valores de la tasa de evaporación del N\(_2\) y de la potencia disipada para cada una de las condiciones experimentales fijadas en la experiencia.}

\end{table}

Para la obtención de la entalpía de vaporización del N\(_2\) se tiene en cuenta
en primer lugar, que la evaporación se produce a presión constante. Por tanto,
como se expuso en la ecuación \ref{eq:orgb37ddea}, en el balance energético solo es
necesario considerar la tasa de evaporación, el incremento de entalpía, las
pérdidas por unidad de tiempo y la potencia disipada. Respecto al término
\(\dot{Q}\) \(-\text{que}\) representa el intercambio de calor por unidad de tiempo
entre el medio y el nitrógeno \(\text{líquido}-\) puede considerarse constante
durante todo el experimento puesto que las temperaturas del sistema y del
ambiente no varían. Por tanto, los datos de la tabla \ref{tab:orgc2c24bf} pueden ajustarse
linealmente a la ecuación:
\begin{equation}
\label{eq:org1c867f9}
\dot{m} = -\frac{1}{\Delta h} \dot{W} + \frac{\dot{Q}}{\Delta h}.
\end{equation}
De nuevo, pueden extraerse del archivo \hyperref[orgd5a5a97]{\emph{fit.log}} generado por \emph{gnuplot} los
parámetros del ajuste con su respectivas incertidumbres. La figura \ref{org39700a4}
representa los conjuntos de datos y su recta de regresión.

\phantomsection
\label{orgd5a5a97}
\begin{verbatim}
Final set of parameters            Asymptotic Standard Error
=======================            ==========================
dh              = 173.157          +/- 9.45         (5.458%)
Q               = 1.79859          +/- 0.1819       (10.11%)
\end{verbatim}

Por tanto, los valores estimados para el incremento de entalpía de vaporización
del nitrógeno, \(\Delta h\), y las pérdidas por unidad de tiempo, \(\dot{Q}\), son:
\begin{equation}
\label{eq:org3d2b541}
\Delta h = (173 \pm 9)\, \text{J g}^{-1}.
\end{equation}
\begin{equation}
\label{eq:org8706534}
\dot{Q} = (1.80 \pm 0.18)\, \text{J s}^{-1}.
\end{equation}

\begin{figure}[h!]
\begin{center}
  \input{entalpiaVaporizacion}
\end{center}
\caption{\label{org39700a4}Representación de los valores obtenidos para las tasas de evaporación, \(\dot{m}\), frente a la potencia disipada, \(\dot{W}\). Ambas magnitudes pueden relacionarse linealmente mediante la ecuación \ref{eq:org1c867f9} de conservación de la energía. A partir del ajuste pueden estimarseo los valores del incremento de entalpía de vaporización del N\(_2\) y de las pérdidas por unidad de tiempo. En rojo se representan las barras de error de los datos.}
\end{figure}

\section{Discusión}
\label{sec:org05bcaaa}

Conceptualmente, los resultados obtenidos concuerdan con lo esperado. La
entalpía de vaporización del nitrógeno es positiva. El nitrógeno líquido
absorbe energía para romper las fuerzas intermoleculares. La vaporización es
un proceso endotérmico. Por otra parte, el nitrógeno líquido se encuentra a
una temperatura inferior a la del ambiente, por lo que absorbe calor de este.
Como ya se ha expuesto, el ritmo del intercambio de calor puede considerarse
constante durante toda la experiencia, por lo que la cantidad de calor
absorbido es cada vez mayor. Este último hecho respalda que las pérdidas
por unidad de tiempo sean positivas.

Cuantitativamente, el valor obtenido para la entalpía de vaporización del
nitrógeno presenta un error relativo del \(13\%\) respecto del valor de referencia
a presión atmosférica (\(199.176\) J g\(^{-1}\)) extraído de \autocite{Lemmon2026}.
Si se analiza la figura \ref{orgba77be9} se observa que la serie de datos
correspondiente a la intensidad de corriente de \(0.45\) A muestra la mayor
dispersión \(-\text{en}\) particular, un \(R^2\) \(\text{menor}-\). Los puntos
correspondientes a las primeras mediciones presentan un comportamiento más
alejado del ajuste lineal. Probablemente esto sea debido a que la evaporación
del N\(_2\) no fuese aún estacionaria en el momento de comenzar la toma de datos.

Si se repite el ajuste de la figura \ref{org39700a4} excluyendo el punto
correspondiente a la intensidad de corriente de \(0.45\) A, se obtiene un
nuevo valor para la entalpía de vaporización del nitrógeno:
\begin{equation*}
\Delta \hat{h} = (190 \pm 3)\, \text{J g}^{-1},
\end{equation*}
cuyo error relativo es del \(5\%\).

\section{Conclusiones}
\label{sec:org53f67c6}

La práctica de determinación de la entalpía de vaporización del nitrógeno
constituye una experiencia con un alto valor académico gracias a la calidad
de sus resultados y a que permite trabajar con sustancias a bajas temperaturas.
Conviene considerar, a la hora de reproducir la experiencia, los tiempos de
espera para que el nitrógeno líquido hierva estacionariamente y los
resultados se ajusten mejor a las hipótesis formuladas.

Otro aspecto donde se han detectado una posibilidad mejora es en la toma de
datos. La anotación cada \(30\) s de las medidas de tres instrumentos
\(-\text{granatario}\), amperímetro y \(\text{voltímetro}-\) introduce un factor de tiempo
de reacción y la necesidad de tener que apuntar simultáneamente tres valores.
La automatización de las lecturas de los instrumentos de medida podría
minimizar los errores introducidos.

\newpage
\printbibliography[heading=bibnumbered]

\includepdf[pages=-]{hojaResultados.pdf}
\end{document}
